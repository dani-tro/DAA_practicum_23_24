\documentclass[12pt]{article}

\usepackage{titlesec}
\titleformat
{\section} % command
[display] % shape
{\normalfont\fontsize{14}{14}\sffamily\bfseries} % format
{} % label
{1ex} % sep
{
    \color{cyan}\rule{0.7\textwidth}{6pt}\vspace{2pt}\newline
    \color{black}
} % before-code

\titleformat
{\subsection} % command
[display] % shape
{\normalfont\fontsize{14}{14}\sffamily\bfseries} % format
{} % label
{1ex} % sep
{
    \color{cyan}\rule{0.5\textwidth}{3pt}\vspace{5pt}\newline
    \color{black}\hspace{11pt}
} % before-code
\titlespacing*{\subsection}{0pt}{0ex}{0em}

\setlength{\parskip}{0.5em}

%https://www.latextemplates.com/template/lachaise-assignment

%---DOCUMENT MARGINS---
\usepackage{geometry} % Required for adjusting page dimensions and margins
\geometry{
	paper=a4paper, % Paper size, change to letterpaper for US letter size
	top=4.25cm, % Top margin
	bottom=3cm, % Bottom margin
	left=2.5cm, % Left margin
	right=2.5cm, % Right margin
	headheight=3cm, % Header height
	%footskip=1.5cm, % Space from the bottom margin to the baseline of the footer
	headsep=0.8cm, % Space from the top margin to the baseline of the header
	%showframe, % Uncomment to show how the type block is set on the page
}

\usepackage{cmap}
\usepackage[utf8]{inputenc}
\usepackage[T2A]{fontenc}
\usepackage[bulgarian,english]{babel}

\usepackage{amsmath}
\usepackage{makecell}
\usepackage{tabulary}


\usepackage{indentfirst}

\usepackage{fancyhdr}
\pagestyle{fancy}
%\setlength{\headheight}{35pt}

\usepackage{background}
\backgroundsetup{contents={\includegraphics[width=28cm,height=3.45cm]{./structure/header_background.png}},scale=1,placement=top,opacity=0.8}

\fancyhead{}
%\fancyhead[L]{\includegraphics[height=2cm]{ruo.jpg}}
%\fancyhead[R]{\includegraphics[height=2cm]{sap.png}}

\fancyhead[L]{\begin{minipage}{\textwidth}\flushleft\includegraphics[width=2.8cm]{./structure/ruo.png} \vspace{10pt} \end{minipage}}
\fancyhead[R]{\begin{minipage}{\textwidth}\flushright\includegraphics[width=2.8cm]{./structure/sap.png}\end{minipage}}

\fancyhead[C]{\begin{minipage}{\textwidth}\color{white}\centering\small\bf{\fontfamily{cmss}\selectfont
\textit{ОТКРИТО ПЪРВЕНСТВО НА СОФИЯ ПО ИНФОРМАТИКА}\\\
\textit{15 ноември 2021 г.}\\
\textit{Група C, 7–8 клас} \vspace{-10pt}} \end{minipage}}
\renewcommand{\headrulewidth}{0pt}

\fancyheadoffset[L,R]{1.5cm}

\pagenumbering{gobble}

\usepackage{graphicx}
\usepackage{wrapfig}
\sloppy

\begin{document}

\noindent\textcolor{cyan}{\rule{10cm}{1.5mm}}
\section*{\hspace{0.61cm}Задача C?. Кули}

%\begin{wrapfigure}{r}{0.25\textwidth}
%\centering
%\includegraphics[width=0.25\textwidth]{olimpiici.png}
%\end{wrapfigure}

Напоследък Алекс доста скучаел, седейки само пред компютъра, и за да преодолее жаждата за сън, решил да си измисли някаква игра, каквато и да е! Първото нещо, което му се мярнало пред погледа, била купчината от гумички, всяка от които с форма на куб $1cm\times1cm\times1cm$. И ето, че идеята веднага се родила. Той искал да ги нареди в редица от кули, така че всяка кула да е съставена от някакъв положителен брой гумички. Също така, всяка кула трябвало да е с височина или строго по-малка от тези на двете ѝ съседни, или строго по-голяма (ако кулата е крайна се гледа само единият ѝ съсед). Разбира се едно такова нареждане е доста лесно за съставяне. На Алекс обаче му било интересно колко са всички възможни начини, по които може да ги подреди, че да е спазено условието. Включете се в неговата кауза срещу скуката, като напишете програма \textbf{towers}, която по даден брой гумички, да намира броя на всички възможни редици от кули от описания вид. Тъй като той може да е доста голям, изведете остатъка му при деление с 1000000007.

\begin{center}
\includegraphics[scale=0.7]{towers_example_1}
\end{center}

\noindent\textcolor{cyan}{\rule{8cm}{1.0mm}}
\subsection*{\hspace{0.61cm}Вход}
На единствения ред на стандартния вход се въвежда едно число $S$ - броят на гумичките.

\noindent\textcolor{cyan}{\rule{8cm}{1.0mm}}
\subsection*{\hspace{0.61cm}Изход}
На един ред изведете едно неотрицателно число - търсения брой начини по модул $10^9+7$.

\noindent\textcolor{cyan}{\rule{8cm}{1.0mm}}
\subsection*{\hspace{0.61cm}Ограничения}
$1 \leq S \leq 5000$
\\ \\ \\ \\

\noindent\textcolor{cyan}{\rule{8cm}{1.0mm}}
\subsection*{\hspace{0.61cm}Подзадачи}
\begin{tabular}{ |c|c|c| }
\hline
\rule{0pt}{2.2ex} \textbf{Подзадача} & \textbf{Точки} & $S$\\ 
\hline
\rule{0pt}{2.2ex} \textbf{1} & 20 & $\leq 10$\\ 
\hline
\rule{0pt}{2.2ex} \textbf{2} & 15 & $\leq 40$\\ 
\hline
\rule{0pt}{2.2ex} \textbf{3} & 10 & $\leq 100$\\ 
\hline
\rule{0pt}{2.2ex} \textbf{4} & 15 & $\leq 500$\\  
\hline
\rule{0pt}{2.2ex} \textbf{5} & 30 & $\leq 2000$\\  
\hline
\rule{0pt}{2.2ex} \textbf{6} & 10 & $\leq 5000$\\  
\hline
\end{tabular}
\newline\newline
\indent \textit{Точки за дадена подзадача се получават ако всички тестове преминат успешно.}

\noindent\textcolor{cyan}{\rule{8cm}{1.0mm}}
\subsection*{\hspace{0.61cm}Пример}
\begin{tabular}{ |c|c|c| } 
\hline
\rule{0pt}{2.2ex}\textbf{Вход} & \textbf{Изход} & \textbf{Обяснение} \\ 
\hline
\rule{0pt}{2.2ex}
\fontfamily{cmss}\selectfont 
\makecell[tl]{6}
& \fontfamily{cmss}\selectfont \makecell[tl]{12 \\} & \makecell[tl]{ Всички възможни редици са: \\ \{1,2,1,2\} \{2,1,2,1\} \{1,3,2\} \{2,1,3\} \{2,3,1\} \\ \{3,1,2\} \{1,4,1\} \{2,4\} \{4,2\} \{1,5\} \{5,1\} \{6\}  \\} \\
\hline
\rule{0pt}{2.2ex}
\fontfamily{cmss}\selectfont 
\makecell[tl]{20}
& \fontfamily{cmss}\selectfont \makecell[tl]{6949 \\} & \makecell[tl]{} \\
\hline

\end{tabular}
	
\end{document}






