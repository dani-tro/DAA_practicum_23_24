\documentclass[12pt]{article}
\newcommand{\bottomMargin}{2cm}

\newcommand{\header}{
	ДИЗАЙН И АНАЛИЗ НА АЛГОРИТМИ - \\ ПРАКТИКУМ\\
	\vspace{0.1cm}
        Летен семестър, 2024 г., домашно\\
	\vspace{0.1cm}
}
\newcommand{\tl}{$1$ сек.}
\newcommand{\ml}{$512$ MB}

%---DOCUMENT MARGINS---
\usepackage{geometry} % Required for adjusting page dimensions and margins
\geometry{
	paper=a4paper, % Paper size, change to letterpaper for US letter size
	top=4cm, % Top margin
	bottom=2cm, % Bottom margin
	left=2cm, % Left margin
	right=2cm, % Right margin
	headheight=3cm, % Header height
	%footskip=1.5cm, % Space from the bottom margin to the baseline of the footer
	headsep=0.5cm, % Space from the top margin to the baseline of the header
	%showframe, % Uncomment to show how the type block is set on the page
}
\usepackage{cmap}

\usepackage[T2A]{fontenc}
\usepackage[bulgarian]{babel}
\usepackage{fontspec}
\setmainfont{Times New Roman}
\setsansfont{Times New Roman}
\setmonofont{Courier New}
\usepackage[math-style=TeX]{unicode-math}
\setmathfont{Latin Modern Math}

\usepackage[nobottomtitles*]{titlesec}
\titleformat
{\section} % command
{\normalfont\fontsize{14}{14}\sffamily\bfseries} % format
{} % label
{0pt} % sep
{} % before-code
\titlespacing{\section}{0pt}{0em}{0em}
\newcommand{\problem}[1]{%
	\section[#1]{#1 \fontsize{12}{12}{\hfill{\normalfont{
				\emoji{hourglass-not-done}\tl \space\space \emoji{floppy-disk} \ml}}}}
}
\usepackage[dvipsnames]{xcolor}
\titleformat
{\subsection} % command
{\fontsize{14}{14}\itshape} % format
{} % label
{0pt} % sep
{} % before-code
[\vspace{-1em}{\color{LimeGreen}\rule{0.2\textwidth}{0.2em}}\vspace{-0.7em}] % after-code
\titlespacing{\subsection}{0pt}{0.5em}{0em}

\setlength{\parskip}{0.5em}
\setlength{\parindent}{24pt}
\sloppy

\usepackage{fancyhdr}
\pagestyle{fancy}
\usepackage{setspace}
\fancyhead[L]{
	\begin{minipage}{\textwidth}
		\includegraphics[width=2.5cm]{./logo.jpg}
	\end{minipage}
}
\fancyhead[C]{
	\begin{minipage}{\textwidth}
		\centering\large\bf{\header}
		\vspace{-0.35cm}
	\end{minipage}
}
\fancyhead[R]{}
\renewcommand{\headrulewidth}{0cm}
\fancyheadoffset[L,R]{1cm}
\usepackage{lastpage}
\fancyfoot[C]{\thepage\ / \pageref*{LastPage}}

\raggedbottom

\usepackage{amsmath}
\usepackage{stmaryrd}

\usepackage{graphicx}
\graphicspath{{./}}
\usepackage[export]{adjustbox}
\usepackage{wrapfig}
\makeatletter
\patchcmd\WF@putfigmaybe{\lower\intextsep}{}{}{\fail}
\AddToHook{env/wrapfigure/begin}{\setlength{\intextsep}{0pt}}
\makeatother
\usepackage[inkscapearea=page,inkscapepath=./svg-inkscape]{svg}
\svgpath{{./}}

\usepackage{makecell}
\usepackage{tabularray}
\AtBeginEnvironment{table}{\vspace{-0.2cm}}
\AtEndEnvironment{table}{\vspace{-0.2cm}}
\usepackage{float}
\usepackage{placeins}
\usepackage{caption}
\captionsetup[table]{
	skip=0.25em,font=it,
	singlelinecheck=false,justification=justified,indention=-24pt,
	margin={24pt, 0pt}
}

\usepackage{enumitem}
\setlist{itemsep=-0.4em,leftmargin=\parindent,topsep=-\parskip}
\newcommand{\tabitem}{\indent~~\llap{\textbullet}~~}

\usepackage{hyperref}
\hypersetup{
	colorlinks=true,
	citecolor=blue,
	linkcolor=blue,
	urlcolor=cyan,
}

\usepackage{emoji}

\begin{document}
\section{Задача H5. СКОРОСТИ}

Пътната мрежа в държавата Nogova се състои от кръстовища и пътища между тях. Наскоро управата реши да въведе задължителна скорост за движение по по-важните пътни артерии. Това означава, че в началото на всяка пътна отсечка ще бъде поставен знак, указващ с каква точно скорост трябва да се движат автомобилите в конкретната отсечка. 

Поставянето на знаците вече започна, но тъй като проектът е съвсем нов, някои пътища в момента са без знак. За да не блокират придвижването в страната, властите разрешават преминаването през пътищата без знаци, но при преминаване по такъв път, шофьорите трябва да запазят скоростта на движение, която са поддържали на предишната пътна отсечка, ако такава има, и 70 км/ч в противен случай.

Вашата цел е по зададена схема на пътната мрежа в страната да намерите най-краткото време, за което можете да достигнете до дадено кръстовище. Приемаме, че ускорението или забавянето при срещане на пътен знак за ограничаване на скоростта се случва мигновено.

\subsection{Вход}

На първия ред на стандартния вход ще бъдат зададени целите числа $N$, $M$ и $S$, където $N$ е броят на кръстовищата, номерирани от $0$ до $N-1$, $M$ е броят на пътищата и $S$ е номерът на
кръстовището, което е крайната дестинация на нашето пътуване(започваме от кръстовище 0). На 
всеки от следващите $M$ реда ще бъдат зададени  четири цели числа $A_i$, $B_i$, $V_i$ и $L_i$, описващи $i$-тия път. 
Числата $A_i$ и $B_i$ описват съответно началното и крайното кръстовища на пътния(еднопосочен) сегмент, $V_i$ е задължителната скорост на движение по сегмента, а $L_i$ е неговата дължина. Ако $V_i$ е нула, това означава, че знакът за ограничение
на скоростта за този сегмент все още не е поставен и превозните средства не трябва да променят скоростта си на движение.


\subsection{Изход}

На единствения ред на стандартния изход изведете последователност от цели числа,
описваща номерата на кръстовищата, през които преминава най-бързият възможен маршрут от
кръстовище $0$ до крайната дестинация $S$. Кръстовищата трябва да са изброени в точния ред, в
който ги обхожда избраният маршрут, започвайки с $0$ и завършвайки със $S$. Входните данни са
подбрани така, че да няма повече от един най-бърз маршрут.

\subsection{Ограничения}

\vspace{0.1em}
\begin{itemize}
	\item $2 \leq N \leq 150$
    \item $1 \leq M \leq 25000$
	\item $0 \leq A_i, B_i < N$
    \item $0 \leq V_i \leq 500$
    \item $1 \leq L_i \leq 500$
\end{itemize}

\subsection{Примери}

\begin{table}[ht]
	\begin{tblr}{|X[17,l]|X[17,l]|}
		\hline
		\textbf{Вход} & \textbf{Изход}\\
		\hline
        \texttt{6 8 1\\
                    0 1 30 90\\
                    0 2 0 70\\
                    2 3 500 1\\
                    3 1 1 300\\
                    1 0 500 1\\
                    0 4 0 70\\
                    4 5 150 1\\
                    5 1 0 150\\
                    } & 
            \texttt{0 4 5 1}\\
		\hline
		\texttt{6 15 1\\
                    0 1 25 68\\
                    0 2 30 50\\
                    0 5 0 101\\
                    1 2 70 77\\
                    1 3 35 42\\
                    2 0 0 22\\
                    2 1 40 86\\ 
                    2 3 0 23\\
                    2 4 45 40\\ 
                    3 1 64 14\\
                    3 5 0 23\\
                    4 1 95 8\\
                    5 1 0 84\\
                    5 2 90 64\\
                    5 3 36 40\\
                    } & 
            \texttt{0 5 2 3 1}
		
		\\
		\hline
	\end{tblr}
\end{table}



\end{document}
