\documentclass[12pt]{article}
\newcommand{\bottomMargin}{2cm}

\newcommand{\header}{
	ДИЗАЙН И АНАЛИЗ НА АЛГОРИТМИ - \\ ПРАКТИКУМ\\
	\vspace{0.1cm}
        Летен семестър, 2024 г., домашно\\
	\vspace{0.1cm}
}
\newcommand{\tl}{$2,2$ сек.}
\newcommand{\ml}{$256$ MB}

%---DOCUMENT MARGINS---
\usepackage{geometry} % Required for adjusting page dimensions and margins
\geometry{
	paper=a4paper, % Paper size, change to letterpaper for US letter size
	top=4cm, % Top margin
	bottom=2cm, % Bottom margin
	left=2cm, % Left margin
	right=2cm, % Right margin
	headheight=3cm, % Header height
	%footskip=1.5cm, % Space from the bottom margin to the baseline of the footer
	headsep=0.5cm, % Space from the top margin to the baseline of the header
	%showframe, % Uncomment to show how the type block is set on the page
}
\usepackage{cmap}

\usepackage[T2A]{fontenc}
\usepackage[bulgarian]{babel}
\usepackage{fontspec}
\setmainfont{Times New Roman}
\setsansfont{Times New Roman}
\setmonofont{Courier New}
\usepackage[math-style=TeX]{unicode-math}
\setmathfont{Latin Modern Math}

\usepackage[nobottomtitles*]{titlesec}
\titleformat
{\section} % command
{\normalfont\fontsize{14}{14}\sffamily\bfseries} % format
{} % label
{0pt} % sep
{} % before-code
\titlespacing{\section}{0pt}{0em}{0em}
\newcommand{\problem}[1]{%
	\section[#1]{#1 \fontsize{12}{12}{\hfill{\normalfont{
				\emoji{hourglass-not-done}\tl \space\space \emoji{floppy-disk} \ml}}}}
}
\usepackage[dvipsnames]{xcolor}
\titleformat
{\subsection} % command
{\fontsize{14}{14}\itshape} % format
{} % label
{0pt} % sep
{} % before-code
[\vspace{-1em}{\color{LimeGreen}\rule{0.2\textwidth}{0.2em}}\vspace{-0.7em}] % after-code
\titlespacing{\subsection}{0pt}{0.5em}{0em}

\setlength{\parskip}{0.5em}
\setlength{\parindent}{24pt}
\sloppy

\usepackage{fancyhdr}
\pagestyle{fancy}
\usepackage{setspace}
\fancyhead[L]{
	\begin{minipage}{\textwidth}
		\includegraphics[width=2.5cm]{./logo.jpg}
	\end{minipage}
}
\fancyhead[C]{
	\begin{minipage}{\textwidth}
		\centering\large\bf{\header}
		\vspace{-0.35cm}
	\end{minipage}
}
\fancyhead[R]{}
\renewcommand{\headrulewidth}{0cm}
\fancyheadoffset[L,R]{1cm}
\usepackage{lastpage}
\fancyfoot[C]{\thepage\ / \pageref*{LastPage}}

\raggedbottom

\usepackage{amsmath}
\usepackage{stmaryrd}

\usepackage{graphicx}
\graphicspath{{./}}
\usepackage[export]{adjustbox}
\usepackage{wrapfig}
\makeatletter
\patchcmd\WF@putfigmaybe{\lower\intextsep}{}{}{\fail}
\AddToHook{env/wrapfigure/begin}{\setlength{\intextsep}{0pt}}
\makeatother
\usepackage[inkscapearea=page,inkscapepath=./svg-inkscape]{svg}
\svgpath{{./}}

\usepackage{makecell}
\usepackage{tabularray}
\AtBeginEnvironment{table}{\vspace{-0.2cm}}
\AtEndEnvironment{table}{\vspace{-0.2cm}}
\usepackage{float}
\usepackage{placeins}
\usepackage{caption}
\captionsetup[table]{
	skip=0.25em,font=it,
	singlelinecheck=false,justification=justified,indention=-24pt,
	margin={24pt, 0pt}
}

\usepackage{enumitem}
\setlist{itemsep=-0.4em,leftmargin=\parindent,topsep=-\parskip}
\newcommand{\tabitem}{\indent~~\llap{\textbullet}~~}

\usepackage{hyperref}
\hypersetup{
	colorlinks=true,
	citecolor=blue,
	linkcolor=blue,
	urlcolor=cyan,
}

\usepackage{emoji}

\begin{document}
\section{Задача H7. ЦАРИЦА}

Разполагаме с безкрайна надолу и надясно шахматна дъска, чиито редове са индексирани с естествените числа, започвайки от $0$ от горе надолу и колони - започвайки от $0$ от ляво надясно. Царица се намира в горния ляв ъгъл на дъската - клетка $(0, 0)$ и иска да стигне до клетка $(N, M)$ за точно $K$ на брой хода. Тъй като дъската ни е безкрайна, царицата ни не се движи по обичайните за шаха правила - ако тя се намира в клетка $(x, y)$ за един ход тя може да се придвижи до клетка $(x', y')$, само ако $x \leq x' \leq x + A$ и  $y \leq y' \leq y + B$, където $A$ и $B$ са предварително зададени естествени числа.

Напишете програма \textbf{queen.cpp}, която намира броя различни начини, по които царицата може да постигне целта си.

Тъй като търсеният брой може да е много голям, изведете го по модул 
 $1000000007 (10^9 + 7)$.

\subsection{Вход}

На първия ред на стандартния вход ще бъдат зададени целите числа $N, M, A, B $ и $K$.

\subsection{Изход}

На единствен ред на стандартния изход изведете едно цяло число - броя начини царицата да изпълни целта си.


\subsection{Ограничения}

\vspace{0.1em}
\begin{itemize}
	\item $1 \leq N, M, A, B, K \leq 10000$
\end{itemize}

\subsection{Подзадачи}
\begin{table}[ht]
	\begin{tblr}{|Q[c,m]|Q[c,m]|Q[c,m]|}
		\hline
		\textbf{Подзадача} & \textbf{Точки} & $N, M, A, B, K \leq$ \\
		\hline
		$1$ & $8$ & 8 \\ 
		\hline
		$2$ & $11$ & 50 \\ 
		\hline
		$3$ & $15$ & 300 \\  
		\hline
		$4$ & $20$ & 1000 \\ 
		\hline
		$5$ & $16$ & 3000 \\ 
		\hline
        $6$ & $30$ & 10000 \\ 
        \hline
	\end{tblr}
	\caption*{Точките за дадена подзадача се получават само ако се преминат успешно всички тестове, предвидени за нея.}
\end{table}
\FloatBarrier

\vspace{3.1em}

\subsection{Примери}

\begin{table}[ht]
	\begin{tblr}{|l|l|X[j]|}
		\hline
		\textbf{Вход} & \textbf{Изход} & \textbf{Коментар}\\
		\hline
		\texttt
            {\makecell[lt]{2 3 2 2 2}} & 
            \texttt{6} &
            {Всички поредици от ходове за постигане на целта при този пример са:

(0, 0) \xrightarrow{} (0, 1) \xrightarrow{} (2, 3)

(0, 0) \xrightarrow{} (0, 2) \xrightarrow{} (2, 3)

(0, 0) \xrightarrow{} (1, 1) \xrightarrow{} (2, 3)

(0, 0) \xrightarrow{} (1, 2) \xrightarrow{} (2, 3)

(0, 0) \xrightarrow{} (2, 1) \xrightarrow{} (2, 3)

(0, 0) \xrightarrow{} (2, 2) \xrightarrow{} (2, 3)

}\\
		\hline
		\texttt
            {\makecell[lt]{90 80 70 60 20}} & 
            \texttt{767250473} &
            {В този пример истинският отговор е голям и е изведен само остатъкът му по модул $10^9+7$.} \\
		\hline
	\end{tblr}
\end{table}



\end{document}
