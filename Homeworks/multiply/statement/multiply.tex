\documentclass[12pt]{article}
\newcommand{\bottomMargin}{2cm}

\newcommand{\header}{
	ДИЗАЙН И АНАЛИЗ НА АЛГОРИТМИ - \\ ПРАКТИКУМ\\
	\vspace{0.1cm}
        Летен семестър, 2024 г., домашно\\
	\vspace{0.1cm}
}
\newcommand{\tl}{$0,8$ сек.}
\newcommand{\ml}{$256$ MB}

%---DOCUMENT MARGINS---
\usepackage{geometry} % Required for adjusting page dimensions and margins
\geometry{
	paper=a4paper, % Paper size, change to letterpaper for US letter size
	top=4cm, % Top margin
	bottom=2cm, % Bottom margin
	left=2cm, % Left margin
	right=2cm, % Right margin
	headheight=3cm, % Header height
	%footskip=1.5cm, % Space from the bottom margin to the baseline of the footer
	headsep=0.5cm, % Space from the top margin to the baseline of the header
	%showframe, % Uncomment to show how the type block is set on the page
}
\usepackage{cmap}

\usepackage[T2A]{fontenc}
\usepackage[bulgarian]{babel}
\usepackage{fontspec}
\setmainfont{Times New Roman}
\setsansfont{Times New Roman}
\setmonofont{Courier New}
\usepackage[math-style=TeX]{unicode-math}
\setmathfont{Latin Modern Math}

\usepackage[nobottomtitles*]{titlesec}
\titleformat
{\section} % command
{\normalfont\fontsize{14}{14}\sffamily\bfseries} % format
{} % label
{0pt} % sep
{} % before-code
\titlespacing{\section}{0pt}{0em}{0em}
\newcommand{\problem}[1]{%
	\section[#1]{#1 \fontsize{12}{12}{\hfill{\normalfont{
				\emoji{hourglass-not-done}\tl \space\space \emoji{floppy-disk} \ml}}}}
}
\usepackage[dvipsnames]{xcolor}
\titleformat
{\subsection} % command
{\fontsize{14}{14}\itshape} % format
{} % label
{0pt} % sep
{} % before-code
[\vspace{-1em}{\color{LimeGreen}\rule{0.2\textwidth}{0.2em}}\vspace{-0.7em}] % after-code
\titlespacing{\subsection}{0pt}{0.5em}{0em}

\setlength{\parskip}{0.5em}
\setlength{\parindent}{24pt}
\sloppy

\usepackage{fancyhdr}
\pagestyle{fancy}
\usepackage{setspace}
\fancyhead[L]{
	\begin{minipage}{\textwidth}
		\includegraphics[width=2.5cm]{./logo.jpg}
	\end{minipage}
}
\fancyhead[C]{
	\begin{minipage}{\textwidth}
		\centering\large\bf{\header}
		\vspace{-0.35cm}
	\end{minipage}
}
\fancyhead[R]{}
\renewcommand{\headrulewidth}{0cm}
\fancyheadoffset[L,R]{1cm}
\usepackage{lastpage}
\fancyfoot[C]{\thepage\ / \pageref*{LastPage}}

\raggedbottom

\usepackage{amsmath}
\usepackage{stmaryrd}

\usepackage{graphicx}
\graphicspath{{./}}
\usepackage[export]{adjustbox}
\usepackage{wrapfig}
\makeatletter
\patchcmd\WF@putfigmaybe{\lower\intextsep}{}{}{\fail}
\AddToHook{env/wrapfigure/begin}{\setlength{\intextsep}{0pt}}
\makeatother
\usepackage[inkscapearea=page,inkscapepath=./svg-inkscape]{svg}
\svgpath{{./}}

\usepackage{makecell}
\usepackage{tabularray}
\AtBeginEnvironment{table}{\vspace{-0.2cm}}
\AtEndEnvironment{table}{\vspace{-0.2cm}}
\usepackage{float}
\usepackage{placeins}
\usepackage{caption}
\captionsetup[table]{
	skip=0.25em,font=it,
	singlelinecheck=false,justification=justified,indention=-24pt,
	margin={24pt, 0pt}
}

\usepackage{enumitem}
\setlist{itemsep=-0.4em,leftmargin=\parindent,topsep=-\parskip}
\newcommand{\tabitem}{\indent~~\llap{\textbullet}~~}

\usepackage{hyperref}
\hypersetup{
	colorlinks=true,
	citecolor=blue,
	linkcolor=blue,
	urlcolor=cyan,
}

\usepackage{emoji}

\begin{document}
\section{Задача H6. УМНОЖАВАЙ}

Дадено ни е мултимножество $S=\{a_1, a_2, a_3, \dots, a_N\}$ от $N$ цели положителни числа. Докато в мултимножеството $S$ има поне два различни елемента $a$ и $b$, които \textbf{не} са взаимно прости, можем да ги изтрием и да добавим произведението им $a \cdot b$ към него. Интересуваме се каква е най-малката големина на $S$, която може да се постигне чрез прилагане на дадената операция множество пъти. Напишете програма \textbf{\texttt{multiply}}, отговаряща на въпроса.


\subsection{Вход}

На първия ред на стандартния вход ще бъде зададено цялото число $N$ - мощността на мултимножеството $S$. На втория ред ще бъдат зададени $N$ на брой цели числа - елементите на мултимножеството, в случаен ред.

\subsection{Изход}

На единствен ред на стандартния изход изведете едно цяло число - най-малката големина на $S$, която може да постигне чрез прилагане на дадената операция.


\subsection{Ограничения}

\vspace{0.1em}
\begin{itemize}
	\item $1 \leq N \leq 500~000$
	\item $1 \leq a_i \leq 10^7$
\end{itemize}

\subsection{Подзадачи}
\begin{table}[ht]
	\begin{tblr}{|Q[c,m]|Q[c,m]|Q[c,m]|Q[c,m]|Q[c,m]|Q[c,m]|}
		\hline
		\textbf{Подзадача} & \textbf{Необходими подзадачи} & \textbf{Точки} & $N$ & $a_i$ & \textbf{Други ограничения}\\
		\hline
		$1$ & -- & $0$ & -- & -- & Примерните тестове. \\ 
		\hline
		$2$ & -- & $10$ & $\leq 5.10^5$ & Прости числа & -- \\ 
		\hline
		$3$ & $1$ & $15$ & $\leq 100$ & $\leq 10^4$ & -- \\ 
		\hline
		$4$ & $1, 3$ & $15$ & $\leq 100$ & $\leq 10^7$ & -- \\ 
		\hline
		$5$ & $1, 3, 4$ & $15$ & $\leq 2\ 500$ & $\leq 10^7$ & -- \\
		\hline
        $6$ & $1, 3$ & $30$ & $\leq 10^5$ & $\leq 5.10^5$ & -- \\
        \hline
        $7$ & $1-6$ & $15$ & $\leq 5.10^5$ & $\leq 10^7$ & -- \\
		\hline
	\end{tblr}
	\caption*{Точките за дадена подзадача се получават само ако се преминат успешно всички тестове, предвидени за нея и необходимите подзадачи.}
\end{table}
\FloatBarrier

\subsection{Примери}

\begin{table}[ht]
	\begin{tblr}{|l|l|X[j]|}
		\hline
		\textbf{Вход} & \textbf{Изход} & \textbf{Обяснение на примера}\\
		\hline
		\texttt
            {\makecell[lt]{6\\1 13 5 3 1 5}} & 
            \texttt{5} &
            {Единствено $5$ и $5$ не са взаимно прости числа, така че може да приложим операцията за тях и да получим мултимножеството $\{25, 1, 13, 3, 1\}$, което е с големина $5$. Операцията не може да бъде прилагана повече върху полученото мултимножество.}\\
		\hline
		\texttt
            {\makecell[lt]{5\\5 2 25 7 14}} & 
            \texttt{2} &
            {Първо можем да приложим операцията за $5$ и $25$, след което да приложим операцията за $14$ и $2$ и накрая да приложим операцията за $7$ и $28$. Така накрая ще получим $\{125, 196\}$.} \\
		\hline
	\end{tblr}
\end{table}

\subsection{Забележка}
За решаване на задачата за пълен брой точки може да е нужно използването на решето на Ератостен. За повече информация можете да прочетете подсекциите ,,Решето на Ератостен'' и ,,Факторизация на много числа'' на следната \href{https://www.informatika.bg/lectures/primes}{страница}.

\end{document}
