\documentclass[12pt]{article}
\newcommand{\bottomMargin}{2cm}

\newcommand{\header}{
	ДИЗАЙН И АНАЛИЗ НА АЛГОРИТМИ - \\ ПРАКТИКУМ\\
	\vspace{0.1cm}
        Летен семестър, 2024 г., домашно\\
	\vspace{0.1cm}
}
\newcommand{\tl}{$0,6$ сек.}
\newcommand{\ml}{$256$ MB}

%---DOCUMENT MARGINS---
\usepackage{geometry} % Required for adjusting page dimensions and margins
\geometry{
	paper=a4paper, % Paper size, change to letterpaper for US letter size
	top=4cm, % Top margin
	bottom=2cm, % Bottom margin
	left=2cm, % Left margin
	right=2cm, % Right margin
	headheight=3cm, % Header height
	%footskip=1.5cm, % Space from the bottom margin to the baseline of the footer
	headsep=0.5cm, % Space from the top margin to the baseline of the header
	%showframe, % Uncomment to show how the type block is set on the page
}
\usepackage{cmap}

\usepackage[T2A]{fontenc}
\usepackage[bulgarian]{babel}
\usepackage{fontspec}
\setmainfont{Times New Roman}
\setsansfont{Times New Roman}
\setmonofont{Courier New}
\usepackage[math-style=TeX]{unicode-math}
\setmathfont{Latin Modern Math}

\usepackage[nobottomtitles*]{titlesec}
\titleformat
{\section} % command
{\normalfont\fontsize{14}{14}\sffamily\bfseries} % format
{} % label
{0pt} % sep
{} % before-code
\titlespacing{\section}{0pt}{0em}{0em}
\newcommand{\problem}[1]{%
	\section[#1]{#1 \fontsize{12}{12}{\hfill{\normalfont{
				\emoji{hourglass-not-done}\tl \space\space \emoji{floppy-disk} \ml}}}}
}
\usepackage[dvipsnames]{xcolor}
\titleformat
{\subsection} % command
{\fontsize{14}{14}\itshape} % format
{} % label
{0pt} % sep
{} % before-code
[\vspace{-1em}{\color{LimeGreen}\rule{0.2\textwidth}{0.2em}}\vspace{-0.7em}] % after-code
\titlespacing{\subsection}{0pt}{0.5em}{0em}

\setlength{\parskip}{0.5em}
\setlength{\parindent}{24pt}
\sloppy

\usepackage{fancyhdr}
\pagestyle{fancy}
\usepackage{setspace}
\fancyhead[L]{
	\begin{minipage}{\textwidth}
		\includegraphics[width=2.5cm]{./logo.jpg}
	\end{minipage}
}
\fancyhead[C]{
	\begin{minipage}{\textwidth}
		\centering\large\bf{\header}
		\vspace{-0.35cm}
	\end{minipage}
}
\fancyhead[R]{}
\renewcommand{\headrulewidth}{0cm}
\fancyheadoffset[L,R]{1cm}
\usepackage{lastpage}
\fancyfoot[C]{\thepage\ / \pageref*{LastPage}}

\raggedbottom

\usepackage{amsmath}
\usepackage{stmaryrd}

\usepackage{graphicx}
\graphicspath{{./}}
\usepackage[export]{adjustbox}
\usepackage{wrapfig}
\makeatletter
\patchcmd\WF@putfigmaybe{\lower\intextsep}{}{}{\fail}
\AddToHook{env/wrapfigure/begin}{\setlength{\intextsep}{0pt}}
\makeatother
\usepackage[inkscapearea=page,inkscapepath=./svg-inkscape]{svg}
\svgpath{{./}}

\usepackage{makecell}
\usepackage{tabularray}
\AtBeginEnvironment{table}{\vspace{-0.2cm}}
\AtEndEnvironment{table}{\vspace{-0.2cm}}
\usepackage{float}
\usepackage{placeins}
\usepackage{caption}
\captionsetup[table]{
	skip=0.25em,font=it,
	singlelinecheck=false,justification=justified,indention=-24pt,
	margin={24pt, 0pt}
}

\usepackage{enumitem}
\setlist{itemsep=-0.4em,leftmargin=\parindent,topsep=-\parskip}
\newcommand{\tabitem}{\indent~~\llap{\textbullet}~~}

\usepackage{hyperref}
\hypersetup{
	colorlinks=true,
	citecolor=blue,
	linkcolor=blue,
	urlcolor=cyan,
}

\usepackage{emoji}

\begin{document}
\section{Задача H8. БИЛЕТИ}

Имаме редица от $N$ билетчета, на всяко от които е записано число $a_i$. Докато в редицата ни има поне $3$ числа, можем да прилагаме следната операция: избираме билетче, което не е първото или последното в редицата, и го премахваме от редицата. При всеки такъв ход получаваме точки, които се изчисляват като умножим числото, записано на избраното билетче, със сбора на числата, записани на съседните му две билетчета.

Напишете програма \textbf{tickets.cpp}, която намира най-големия сумарен брой точки, който може да се получи при прилагане на горната операция.

\subsection{Вход}

На първия ред на стандартния вход ще бъде зададено цялото число $N$ - броят на билетчетата в редицата.
На втория ред на стандартния вход ще бъдат зададени $N$ цели числа $a_1, a_2, \dots, a_n$ - числата, записани на билетчетата.

\subsection{Изход}

На единствен ред на стандартния изход изведете едно цяло число - максималния сумарен брой точки, който може да се получи при прилагане на операцията.


\subsection{Ограничения}

\vspace{0.1em}
\begin{itemize}
	\item $3 \leq N \leq 700$
    \item $1 \leq a_i \leq 1000$
\end{itemize}

\subsection{Примери}

\begin{table}[ht]
	\begin{tblr}{|l|l|}
		\hline
		\textbf{Вход} & \textbf{Изход} \\
		\hline
		\texttt
            {\makecell[lt]{4 \\ 3 7 4 6}} & 
            \texttt{115} \\
		\hline
		\texttt
            {\makecell[lt]{6 \\ 1 3 6 2 5 4}} & 
            \texttt{129} \\
		\hline
        \texttt
            {\makecell[lt]{7 \\ 5 2 1 7 4 3 3}} & 
            \texttt{159} \\
		\hline
	\end{tblr}
\end{table}



\end{document}
